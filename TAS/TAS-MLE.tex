%-----------------------------------------------%
\section{Maximum Likelihood}

In statistics, a likelihood function (often simply the likelihood) is a function of the parameters of a statistical model. 
The likelihood of a set of parameter values, θ, given outcomes x, is equal to the probability of those 
observed outcomes given those parameter values, that is $\mathcal{L}(\theta |x) = P(x | \theta)$.

%Page 7 of TAS submisssion

\[ \mathcal{l} (\mu_d,\sigma^2_D,\sigma^2_S,\rho_{DS}) = \frac{-n}{2}\mbox{log}(2\pi)  - \frac{n^2}{2}\mbox{log}(\sigma^2_D) 
\frac{n(n-1)}{2}\mbox{log}(1-\rho^2_{DS})

\]

%-----------------------------------------------%
\section{Components}
\newline
\textbf{Part 1} $\sigma_s$ \\

\[ \frac{ \partial l}{\partial \sigma_s}  = \frac{-\rho}{1-\rhos^2} \sum \left[ \frac{\hat{d}_i}{\sigma_D}  -\rho \frac{\hat{s}_i}{\sigma_s} \right] \frac{\hat{s}_i}{\sigma^2_s} \]

Let this = 0 to determine Max or Min.


%%% CHECK THIS ONE


%------------------------------------------------%
\newline
\textbf{Part 2} $\sigma_D$ \\






Let this = 0 to determine Max or Min.
%------------------------------------------------%
\newline
\textbf{Part 3}: $\rho_{DS}$

\[ \frac{\partial \mathcal{l}}{\partial \rho_{DS}} 
  = \frac{\partial }{\partial \rho_{DS}}  \left[ \frac{n(n-1)}{2} \mbox{log}(1-\rho^2_{DS} \right]  \]
  
Letting $\mathcal{l} = y$ and $\rho_{DS}=x$ we can express this as
\[ y = log(1-x^2)\]
Using the Chain Rule to differentiate
\[ dy/dx = 2x. \frac{1}{1-x^2} = \frac{2x}{1-x^2} \] 

Trial Solution : 2x = 0, hence x=0. Therefore $\rho^2_{DS}=0$.



